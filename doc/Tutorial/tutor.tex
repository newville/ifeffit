%%  tutor.tex   -*-latex-*-
%%
\documentclass[11pt]{article}
\usepackage{fancybox,fancyhdr,times}
\newcommand{\Vers}{{1.2.6}}
\newcommand{\update}{July 07, 2004}


%%#Macros%
\include{ifmacs_pdf}

 \setlength{\oddsidemargin}{0.25in}
 \setlength{\evensidemargin}{-0.25in}
 \setlength{\topmargin}{-0.25in}
 \setlength{\headsep}{0.375in}
 \setlength{\textheight}{9.00in}
 \setlength{\textwidth}{5.75in}

 
 \pagestyle{fancy} \fancyhf{}
 \addtolength{\headwidth}{\marginparwidth}
 \addtolength{\headheight}{2pt}
 \renewcommand{\headrulewidth}{0.7pt} 
 \rhead{\slshape\thepage}\lhead{\bfseries\leftmark}
 \cfoot{}
  


 \definecolor{LightGrey}{gray}{0.675}
 \definecolor{HalfGrey}{gray}{0.5}
 \newcommand{\greyline}{{\color{LightGrey}{\rule{\linewidth}{0.75mm}}\par}}
 \newcommand{\ThickGreyLine}{{\color{HalfGrey}{\rule{\linewidth}{5pt}}\par}}

 \newcommand{\program}[1]{{{\scshape{#1}}}}
 \newcommand{\uwxafs}{\program{uwxafs3.0}}
 \newcommand{\feff}{\program{feff}}
 \newcommand{\feffit}{\program{feffit}}
 \newcommand{\autobk}{\program{autobk}}
 \newcommand{\atoms}{\program{atoms}}
 \newcommand{\ifeffit}{\program{ifeffit}}
 \newcommand{\gifeffit}{\program{g.i.feffit}}
 \newcommand{\athena}{\program{athena}}
 \newcommand{\artemis}{\program{artemis}}

 \newcommand{\XAIBook}{\textsl{XAFS Analysis with {\ifeffit} }}
 \newcommand{\IFFtut}{\textsl{The {\ifeffit} Tutorial}}
 \newcommand{\IFFrefg}{\textsl{The {\ifeffit} Reference Guide}}

\newcommand{\cmnd}[1]{{\texttt{#1()}}}
%external file name
\newcommand{\file}[1]{{{\slshape\ttfamily{#1}}}}
  \newcommand{\feffndat}{\file{feffnnnn.dat}}
  \newcommand{\feffbin}{\file{feff.bin}}
  \newcommand{\autobkinp}{\file{autobk.inp}}
  \newcommand{\feffitlog}{\file{feffit.log}}
%math constructions
  \newcommand{\Eo}{{\ensuremath{E_0}}}
  \newcommand{\muE}{{\ensuremath{\mu(E)}}}
  \newcommand{\Del}{{\ensuremath{\Delta \mu_0(E_0)}}}
  \newcommand{\bkg}{{\ensuremath{\mu_{0}(E)}}}
  \newcommand{\bkgk}{{\ensuremath{\mu_{0}(k)}}}
  \newcommand{\CHI}{{\ensuremath{\chi}}}
  \newcommand{\chiE}{{\ensuremath{\chi(E)}}}
  \newcommand{\chik}{{\ensuremath{\chi(k)}}}
  \newcommand{\chir}{{\ensuremath{\tilde\chi(R)}}}
  \newcommand{\chiq}{{\ensuremath{\tilde\chi(k)}}}
  \newcommand{\chisqr}{{\ensuremath{\chi^2}}}
  \newcommand{\redchi}{{\ensuremath{\chi^2_{\nu}}}}

  \newcommand{\Rmax}{{\ensuremath{R_{\rm max}}}  }
  \newcommand{\kmin}{{\ensuremath{k_{\rm min}}}  }
  \newcommand{\kmax}{{\ensuremath{k_{\rm max}}}  }
  \newcommand{\Rbkg}{{\ensuremath{R_{\rm bkg}}}  }
  \newcommand{\Nbkg}{{\ensuremath{N_{\rm bkg}}}  }
  \newcommand{\RIst}{{\ensuremath{R_{\rm 1st}}}  }

  \newcommand{\fp}{\ensuremath{{f'(E)}}}
  \newcommand{\fpp}{\ensuremath{{f''(E)}}}
  \newcommand{\angst}{\ensuremath{{\rm\,\AA}}}
  \newcommand{\iangst}{\ensuremath{{\rm\,\AA^{-1}}}}
  \newcommand{\ie}{{\emph{i.e.}}}
  \newcommand{\eg}{{\emph{e.g.}}}
  \newcommand{\etal}{{\emph{et al.}}}
  \newcommand{\abinitio}{{\emph{ab initio}}}
%%
  \def\delim{\ensuremath{\langle {\rm delimiter} \rangle}}
  \def\rmand{{\rm and\ }}      \def\rmor{{\rm or\ }}
  \def\caret{{{\^\ \kern-0.4em}}}


\newcommand{\MNseealso}[1]{{{\texttt{{#1}}}(Section~\ref{Ch:Command:#1})}}

\newcommand{\indexcmd}[1]{{\index{#1@{{\cmnd{#1}}}}}}
\newcommand{\indexcmds}[2]{\index{#1@{{\cmnd{#1}}}!{#2}}}
\newcommand{\indexvar}[1]{{\index{#1@{{\tt{#1}}}}}}
\newcommand{\indexfil}[1]{{\index{#1@{{\file{#1}}}}}}

\newcommand{\subfunc}[1]{\relax\par\vfill\goodbreak%
\subsection{{\texttt{{#1}}}}{\label{Ch:Command:#1}\indexcmd{#1}}}

\newcommand{\subfuncn}[1]{\subsection{{\texttt{{#1}}}}%
{\label{Ch:Command:#1}\indexcmd{#1}}}


\newcommand{\entryxlabel}[1]{\mbox{{\textsf{#1: }}}}
\newcommand{\entrylabel}[1]{\mbox{{\textsf{#1: }}}\hfil}
\newcommand{\yes}{Y}
\newenvironment{IFFcom}{\begin{list}{}{\vspace{-0.05truein}
      \renewcommand{\makelabel}{\entrylabel}\settowidth{\leftmargin}{97pt}
      \settowidth{\labelwidth}{95pt}\setlength{\rightmargin}{15pt}
      \setlength{\labelsep}{2pt}\setlength{\listparindent}{0pt}
      \setlength{\itemindent}{0pt}\setlength{\itemsep}{-1pt}}}
  {\end{list}}



\makeindex

\begin{document}

\begin{flushright}\vspace*{0.375in}
  \vspace{10mm}{\ThickGreyLine}  \vspace{5mm}
  {\Huge{The {\ifeffit} Reference Guide}}
  \vspace{100mm} {\ }
  
  \thispagestyle{empty}
  \setcounter{page}{0} 
  
  {\Large Matthew Newville}\par
  Consortium for Advanced Radiation Sources\par
  University of Chicago, Chicago, IL\par
  
  \thispagestyle{empty}  \vspace{5mm} {\ThickGreyLine}
  
  {Version {\Vers}}\par
  {\update}\par
  
\end{flushright}

% 


















\include{license}

\section{Introduction}\label{s:intro}

{\ifeffit} is an interactive program for XAFS data analysis.  The main
program runs like a command-line 'shell', in which you enter commands to
process and manipulate data.  {\ifeffit} has a high-level command language,
allowing you to do the complex data manipulation needed for XAFS analysis
(such as background subtraction and Fourier transforms) with simple
commands.

One of the principle features of {\ifeffit} is that it's command-line
functionality can be run in either interactively, from files of commands
(i.e., batch files), or accessed from within other programming and
high-level scripting languages like Tcl, Perl, and Python.  While the
details of how this is done are beyond the scope of this tutorial, most
people actually use {\ifeffit} through one of the GUI programs written
on-top of the basic {\ifeffit} engine, such as {\program{athena}},
{\program{artemis}}, or {\program{sixpack}}.  In this sense, {\ifeffit} is
not a single program, but a family of related programs and libraries using
a common underlying engine.  

This tutorial gives a brief description of the command structure and syntax
of the {\ifeffit} engine, and an overview of the main {\ifeffit}
command-line program.  Other documentation is available at the {\ifeffit}
web site: {\WWWiff}

\subsection{Running {\ifeffit}}
\label{s:running}

Once {\ifeffit} has been installed on your computer, typing {\tt{ifeffit}}
at the system command prompt (or double-clicking on the appropriate icon)
will start the basic {\ifeffit} program.  You should get a set of messages
and a command prompt that looks something like this:
{\small\begin{verbatim}
  Ifeffit 1.2.6  Copyright (c) 2004 Matt Newville, Univ of Chicago
                   command-line shell version 1.1 with GNU Readline
  Ifeffit>
\end{verbatim}}\noindent
At this point, you're ready to start typing {\ifeffit} commands at the
prompt. Try typing
{\small\begin{verbatim}
  Ifeffit> print '1 + 1  = '  1+1

\end{verbatim}}
\noindent
If the result makes sense to you, you're ready to continue.  Now that
you've started {\ifeffit} successfully, the rest of this tutorial describes
{\emph{what}} to type.  To exit {\ifeffit}, you can type {\tt{quit}}.

The main {\ifeffit} program has a friendly shell environment and allows a
simple subset of system-level commands to be executed from within the
command-line program.  On Unix systems, the commands {\tt{ls}} and
{\tt{more}} will give a directory listing and show the contents of a file,
respectively.  On Windows, the command {\tt{dir}} takes the place of
{\tt{ls}}.  Typing {\tt{help}} will give a brief list of the most common
commands, while {\tt{help \emph{<command\_name>}}} will give a little more
of information on the nature and use of the selected command.  The command
{\emph{history buffer}} is accessible through the up- and down-arrow keys,
so that you can scroll through and edit previously executed commands.
Further information about the command-line program on Unix systems is given
in section~{\ref{s:cmdline}}.

%%%%%%%%%%%%%%%%%%%%%%%%%%%%%%%%%%%%%%%%%%%%%%%%%%%%%%%%%%%
%%%
\clearpage
\section{Data, Commands, and Simple Data Manipulation}
\label{s:datatypes}
%%%%%%%%%%%%%%%%%%%%%%%%%%%%%%%%%%%%%%%%%%%%%%%%%%%%%%%%%%%

{\ifeffit} has a simple view of data, and gives you access this data
through high-level commands.  There are three types of data that {\ifeffit}
distinguishes: {\emph{scalars}} contain a single floating point number,
{\emph{arrays}} contains a list or vector of floating point numbers, and
{\emph{strings}} contains a set of text characters.  If you've done any
programming, these data types should be familiar to you.  In keeping with
the language of computer programming, I'll refer to the data in {\ifeffit}
as {\emph{Variables}}.  These are not necessarily the quantities varied in
a fit you might do with your data: I'll call those {\emph{Fitting
    Variables}} when there's room for confusion.

\subsection{Data Types and Naming Conventions}\label{s:datatypes:names}

Variables in {\ifeffit} are named, and you can create, name, and manipulate
your own data variables.  This makes {\ifeffit} a fairly general purpose
calculator and data plotter.  That is, you can type something like this at
the {\ifeffit} command line:

{\small\begin{verbatim}
  Ifeffit> a =  5
  Ifeffit> phi =  (sqrt(a) + 1 ) /2
  Ifeffit> print a, phi
     5.00000000     1.61803399
  Ifeffit> print sin(10)
   -0.544021111
\end{verbatim}}\noindent
Well, that only shows how to use {{scalars}}, not arrays or strings, but it
does show how {\ifeffit} allows you to do simple calculations using syntax
similar to most procedural programming languages. 


{\ifeffit} distinguishes its three data types {\emph{by name}}. This allows
both you and {\ifeffit} to know exactly what kind of data each variable
holds.  Here are the naming rules for the variables:

{\bf{Scalars}} must have names that begin with a letter, ampersand '\&', or
underscore '\_', and then contain letters, numbers, ampersands, and
underscores after that.  The names are not sensitive to case: {\tt{A}} is
the same as {\tt{a}}.


{\bf{Arrays}} have names that always have exactly one dot ('.').  This
gives array names a prefix and suffix.  The prefix of the array name is
associated with the array {\emph{Group}}, which is a simple and effective
way to make several arrays related to each other.  The naming rules for the
prefix or group name are exactly the same as for scalars.  The suffix of
the array name is associated with the array contents.  The naming rules for
the suffix are similar to those for scalars, but relaxed to allow it to
begin with numbers as well as letters, ampersand '\&\, or underscore '\_'.

Thus, 'data.energy' and 'data.xmu' are array names, and are said to be in
the group 'data'.  When we get to discussing commands for doing background
removal, Fourier transforms, and the like, we'll see that arrays created by
{\ifeffit} commands will use the same group name as the input data, which
make it easy to keep a group of data together.  Other valid array names are
'cu.1' and '\_XX\_.001'.

{\bf{Strings}} have names that always begin with a dollar sign '\$', and
contain letters, numbers, ampersands, and underscores after that.  You can
define a string like this:
{\small\begin{verbatim}
  Ifeffit> $string = Gosh, this is easy!
  Ifeffit> print $string
    Gosh, this is easy!
\end{verbatim}}\noindent
In fact, '\$1' and '\$99'' are valid string names, but their use is
discouraged: '\$1' ... '\$9' may be internally overwritten whenever you
invoke a macro, so it's not a good idea to rely on their values.

There's one more thing to note on names for variables.  By convention,
``system variables'' begin with an ampersand '\&' (or '\$\&' for system
strings).  This convention is not enforced in any way, but {\ifeffit} tends
to use such system variables for things that effect its behavior (such as
how output is written to the screen).  Unusual behavior may result if you
write over system variables without knowing what you're doing.

\subsection{Data Manipulation}\label{s:datatypes:manip}

As shown in the brief example at the beginning of the previous section,
working with scalar variables in {\ifeffit} is easy.  You can create and
use variables with normal algebraic syntax:
{\small\begin{verbatim}
  Ifeffit> a   =  5
  Ifeffit> phi =  (sqrt(a) + 1 ) /2
  Ifeffit> x   = pi / phi
  Ifeffit> y   = cos(7*x)
  Ifeffit> print a, phi, x, y
   5.00000000    1.61803399    1.94161104   0.519178666      
\end{verbatim}}\noindent
All scalars are floating point numbers (16 bits of precision).  The usual
mathematical operators (sin, tan, log, exp, coth, and so on) are supported,
and a few operators not often found are supported as well{\footnote{For a
    complete list of operators, consult the Reference Guide}}. The
variables you define can be used anywhere in the mathematical expressions
and assignment statements for other variables.

Manipulating arrays is just as easy as manipulating scalars, though
creating arrays of data is a bit more work.  One very common way to create
arrays is to read them in from data files -- that's covered in the next
section\ref{s:fileio}.  You can also create arrays from scratch using the
built-in functions {\tt{indarr()}}, {\tt{ones()}}, and {\tt{zeros()}}.  For
example
{\small\begin{verbatim}
  Ifeffit> test.index = indarr(10)
\end{verbatim}}\noindent
will create an array with elements (1,2,3,...,10).   You can 
make other evenly spaced arrays:
{\small\begin{verbatim}
  Ifeffit> npts = 100
  Ifeffit> step = 0.01
  Ifeffit> test.index = step * indarr(npts)
\end{verbatim}}\noindent
which will create an array with elements (0.01,0.02,...,1.0). 
(Starting with version 1.0053 you can also say
{\small\begin{verbatim}
  Ifeffit> test.ones = range(0.01,1.0, 0.01)
\end{verbatim}}\noindent
to get the same effect.  The range function takes "start, stop, step" as
it's arguments).  In addition, you can create arrays using 
{\small\begin{verbatim}
  Ifeffit> data.ones = ones(10)
  Ifeffit> data.null = zeros(1000)
\end{verbatim}}
\noindent
which will create first an array with 10 elements, all set to 1:
(1,1,1,...,1) and then an array of 1000 zeros.

Once you have arrays created or read in from data files, manipulating them
is easy:
{\small\begin{verbatim}
  Ifeffit> test.index = indarr(100)
  Ifeffit> test.sqrt  = sqrt( test.index / 10)
\end{verbatim}}\noindent
will fill {\tt{test.sqrt}} with square-roots of the numbers (0.1, 0.2, ....
10.0).  Note that the assignment of {\tt{test.sqrt}} is automatically done
{\emph{element-by-element}}, without looping over elements needed.  In
fact, {\ifeffit} doesn't even allow looping over the elements of an array.


\subsection{Commands and their conventions}
\label{s:datatypes:commands}

The operations you type at the {\ifeffit} command line are interpreted as
commands. In general, {\ifeffit} commands consist of a name, followed by a
set of arguments, usually with a {\emph{keyword/value}} syntax:
{\small\begin{verbatim}
   Ifeffit> command(key= value, key= value, key= value, ...)
\end{verbatim}}
\noindent

The parentheses are optional, but if an opening parenthesis is used just
after the command name, the closing one is required even if that means the
command has to extend over multiple lines.  We'll see that many commands
will become quite long, so this ability will become convenient.  Though
they are optional, I'll use the parentheses when talking about commands, so
you can tell I mean a command when I say {\tt{print()}}.  The keywords are
usually short descriptive names describing some parameter the command may
need.  The value is often a number, but can often be a variable, string, or
even a mathematical expression -- a typical command would look like this:
{\small\begin{verbatim}
   Ifeffit> spline(energy=data.e, xmu =data.xmu, rkbg=1, kweight="2")
\end{verbatim}}
\noindent
The {{keyword/value}} syntax is not universal, and some commands (like the
{\tt{print()}} command shown earlier) take simple lists of arguments
separated by commas.  Some commands even mix lists and {{keyword/value}}
pairs:
{\small\begin{verbatim}
   Ifeffit> command(argument1, argument2, argument2)
   Ifeffit> command(argument1, argument2, key= value, 
                    key= value, key=value)
\end{verbatim}}
\noindent
This may seem a little surprising, especially since in the previous section
we just used  
{\small\begin{verbatim}
   Ifeffit> a = 2
\end{verbatim}}
\noindent
which appears to have no command at all!  The truth is that {\ifeffit} 
always expects a command to be the first word typed at the command line, 
but if the first word is not a known command, it uses the default command
{\tt{def()}} -- short for define, not default -- to  define a variable. 
That is, the above definition was translated to 
{\small\begin{verbatim}
   Ifeffit> def( a = 2 )
\end{verbatim}}
\noindent
As we'll see in the next section, this can have some profound consequences,
so it is often useful to keep in mind that the {\tt{def()}} command is the
default.  I'll continue to use the simpler ``{\tt a = 2}'' syntax
throughout this tutorial, and expect that you will too.

\subsection{Storing Definitions of Scalars and Arrays: 
  {\tt{show()}} and {\tt{def()}}}\label{s:datatypes:setdef}

Because {\ifeffit} is primarily an XAFS modeling program, it is important
to be able to set up both simple and complex models for XAFS path
parameters that can be adjusted during a fit.  To allow a flexible
modeling environment, a principle feature of {\ifeffit} is to allow you to
define Program Variables {\emph{by formula}} and have the values
automatically updated when the values of variables in the formula change.

An example will probably help.  Let's consider the case of entering this
seemingly innocent set of assignments (which, we now know, will use the 
{\tt{def()}} command):
{\small\begin{verbatim}
  Ifeffit> a = 1
  Ifeffit> b = a + 1
  Ifeffit> a = 2
\end{verbatim}}\noindent
What value should {\tt{b}} have: 2 or 3?  In most computer languages and
programs, {\tt{b}} is 2, because the formula for it has not been stored,
only the value at the time of its assignment.  For {\ifeffit} the value
will be 3 -- the formula is stored, not just the value.

The main advantage for this is that you could tell {\ifeffit} that {\tt{a}}
is a fitting variable (by saying {\tt{guess a = 1}}, and use both {\tt{a}}
and {\tt{b}} for parameters in the fitting model.  No matter what value the
fitting engine decides {\tt{a}} should have, {\tt{b}} will always obey the
formula you specified.  When we get to fitting XAFS and non-XAFS data,
you'll find (at least, eventually) this somewhat unique behavior to be very
useful. 

Sometimes, however.  you really want {\tt{b}} to stay as 2, not be
dependent on the future value of {\tt{a}}.  That is, you sometimes want to
turn off the 'store the formula' aspect of {\ifeffit}.  To do this, all you
need to do is use the {\tt{set()}} command as an alternative to {\tt{def()}}:
{\small\begin{verbatim}
  Ifeffit> a = 1
  Ifeffit> set(b = a + 1)
  Ifeffit> a = 2
\end{verbatim}}\noindent
Now the formula for {\tt{b}} will not be saved, and it will remain 2 no
matter how {\tt{a}} changes.
Again, it is often useful to remember: 
\par\indent\indent 
{\bf The default command is {\tt{def()}}, which will save a variables definition.}


\subsection{The {\tt{show()}} command}
\label{s:datatypes:show}

At some point you're going to want to get information back from {\ifeffit}
such 'what exactly {\emph{is}} the value of {\tt{e0}}, anyway?' and 'what
are the names of all the arrays I have?'.  There are two main commands for
showing such information about program variables, and there's no better
place to introduce them then right now.  The first command is
{\tt{show()}}, which will show information about Program Variables and
other {\ifeffit} objects like macros and paths (which we haven't gotten to
yet, but which you'll find useful soon).  To follow the example of the
previous section, we can see the value and definitions of the scalars
{\tt{a}} and {\tt{b}} like this:
{\small\begin{verbatim}
  Ifeffit> a = 2
  Ifeffit> b = a + 1
  Ifeffit> show a
   a              =       2.000000000
  Ifeffit> show b
   b              =       3.000000000 := a+1
\end{verbatim}
}\noindent
Note that not only the values are shown, but also the definitions, where
appropriate.  Although {\tt{a}} wasn't actually {\tt{set()}}, it was
defined as an obviously constant value that didn't depend on any program
variables, so {\feffit} knew to treat it as a {\tt{set}} value.  The
{\tt{show()}} command takes a list of things (scalars, strings, arrays,
etc) to show.  We could have said something like this:
{\small\begin{verbatim}
  Ifeffit> a = 2,  b = a + 1
  Ifeffit> $doc_string = "here is a simple definition"
  Ifeffit> show $doc_string, a, b
   $doc_string    = here is a simple definition
   a              =       2.000000000
   b              =       3.000000000 := a+1
\end{verbatim}
}\noindent   % $
For arrays, {\tt{show()}} doesn't show all the data points, but a one-line
summary of the data:
{\small\begin{verbatim}
  Ifeffit> test.ones = ones(10)
  Ifeffit> test.index = 0.1 * indarr(100)
  Ifeffit> show test.ones, test.index
   test.ones      =   10 pts [ 1.000 : 1.000 ] := ones(10)
   test.index     =  100 pts [0.1000 : 10.00 ] := 0.1*indarr(100)
\end{verbatim}
}\noindent
which shows {\tt{test.index}} to have 100 point, with a minimum value of
0.1 and a maximum value of 10, for example.  As for defined scalars, the
definition is shown after the ``:='' characters.

It's often necessary to show {\emph{all}} the scalars, arrays, or string
variables.  The {\tt{show()}} has several modifiers to tell it to show
entire classes of program variables.  All the modifiers begin with the
{\tt{@}} symbol, so to see the values and definitions of arrays, you'd say
{\tt{show @arrays}}.  To see all the scalars and strings, you'd say
{\tt{show @scalars}} and {\tt{show @strings}}.

If you try {\tt{show @scalars}} or {\tt{show @strings}}, you'll notice
several scalars and strings that you didn't define, but are loaded in to
the program as it starts.  These include {\tt{pi}} and
{\tt{etok}}\footnote{{\tt{etok}} is the value of $2m_e/\hbar^2$ in units on
  $\rm {\AA}^2/eV$, and is useful for converting x-ray energy values to
  photo-electron wavenumber: {\tt{ set kval = sqrt(etok * (energy-e0)) }}}
and several ``system variables'' that begin with a {\&} (or \$\& for system
string variables).  As mentioned at the end of
section~\ref{s:datatypes:names}, these ``system variables'' are used
internally by {\ifeffit}, and though you can change their values, this is
not necessarily recommended.  For the most part these ``system variables''
can be ignored during normal use, and won't be discussed into detail in
this tutorial.

You may also notice that the order the scalars shown by {\tt{show
    @scalars}} is not the same order you put them in.  The order may even
change over time.  This reflects the fact {\ifeffit} tries to manage the
scalars and definitions for its own internal efficiency, and will attempt
to arrange it so that the {\tt{set()}} values are listed before the
{\tt{def()}} values.  Since many commands can change the list of scalars,
the order of listing may change at any time.  The same behavior applies to
arrays: {\ifeffit} will rearrange the list of arrays to suit its own needs
and to try to list the ``constant'' arrays before the defined arrays.

Speaking of array data, it's useful to see information about the groups of
arrays, as analysis threads are generally done according to group. To show
all the arrays in a particular group, you'd say {\tt{show @group=data}},
which will show all the arrays in the group {\tt{data}} as if you had said
{\tt{show data.energy, data.xmu}} and so forth.  To get a list of the array
groups, type {\tt{show @groups}}.

The {\tt{show()}} command can take other ``{\tt{@}}'' modifiers for {\feff}
paths, macros, and fitting variables.  These will be discussed when the
time comes to discuss {\tt{feff}} paths, writing macros, and fitting.  The
{\tt{show()}} command has a modifier to show a brief description of all the
commands: {\tt{show @commands}} will list of all commands.

\subsection{The {\tt{print()}} command}
\label{s:datatypes:print}

We used the {\tt{print()}} command before, but haven't really explained
what it's doing. Unlike the {\tt{show()}} command, which tends to
show information about the data {\emph{type}}, the {\tt{print()}}
command is a more literal command.  That is, {\tt{print e0}} will simply
print the values of the scalar {\tt{e0}}.  No definitions will be shown.
For array data, the entire array will be printed -- hardly ever what you
really want, but sometimes it's necessary.  At it's simplest, then, the
{\tt{print()}} command prints the values of variables listed:

{\small\begin{verbatim} 
  Ifeffit> number = 99.0 
  Ifeffit> print number 
   99.0000000 
  Ifeffit> print pi, number, $doc_string 
   3.14159265 99.0000000 here is a string
\end{verbatim}
}\noindent %$
In addition to this simple behavior, the {\tt{print()}} command can print
literal strings and also evaluate expressions in place.  Thus, you can use
{\tt{print()}} to write simple messages:
{\small\begin{verbatim}
  Ifeffit> print " the square root of ", number, " is ", sqrt(number)
   the square root of    99.0000000  is    9.94987437
  Ifeffit> $descrip = " # of seconds per year"
  Ifeffit> value    = 365*24*60*60
  Ifeffit> print  $descrip, " = ", value
   # of seconds per year =    31536000.0
\end{verbatim}
}\noindent
In addition to the {\tt{print()}} command, there's also an {\tt{echo()}}
command that simply prints a string:
{\small\begin{verbatim} 
  Ifeffit> echo "Hi Mom!"
   Hi Mom!
\end{verbatim}
}\noindent %$
This is not incredibly useful when typing at the command line, but does
become useful when you load files of {\ifeffit} commands, as we'll see
next. 

\subsection{Command Files}
\label{s:datatypes:load}

Typing at the command line is all well and good until you have to do it the
third or fourth time, at which point it becomes pretty tedious.  More
importantly, it's not convenient for processing lots of data.  For that,
you'd like to be able to edit text files of {\ifeffit} commands and run
them all at once.  You can.   A file of commands can be loaded with the
{\tt{load()}} command, which will run through all the commands in the
file.  A command file {\file{show\_bkg.iff}} that looks like this
{\small{
%VerbSBox%
\begin{VerbSBox}
 # File  show_bkg.iff 
 read_data(file=Cu.dat, type=raw, group= cu)
 cu.energy  = cu.1 * 1000.0
 cu.xmu     = ln(cu.2 / cu.3)
 spline(energy = cu.energy, xmu = cu.xmu, 
        rbkg=1.1, kweight=1., kmin=0)

 plot(cu.energy, cu.xmu)
 plot(cu.energy, cu.bkg, xmin=8850, xmax=9300,  
                color=red)
 #
\end{VerbSBox}
%VerbSBox%
}}\noindent
can be loaded as
{\small\begin{verbatim} 
  Ifeffit> load show_bkg.iff
\end{verbatim}
}\noindent
More than that, the history mechanism of the {\tt{ifeffit}} command-line
program saves a list of the 500 most recent {\ifeffit} commands run to the
file {\file{.ifeffit\_hist}} in your home directory.  This file can be used
as a starting point for creating and editing command files.  These
topics will be discussed further in section~{\ref{s:log-save}}



%%%%%%%%%%%%%%%%%%%%%%%%%%%%%%%%%%%%%%%%%%%%%%%%%%%%%%%%%%%
%%%
\clearpage
\section{Reading Arrays from Data Files}\label{s:fileio}
%%%%%%%%%%%%%%%%%%%%%%%%%%%%%%%%%%%%%%%%%%%%%%%%%%%%%%%%%%%

{\ifeffit} reads ASCII files with data listed in columns, delimited by
whitespace (blanks or tabs).  The data are stored in {\ifeffit} arrays, so
that they are ready for immediate manipulation, plotting, and analysis.
The {\tt{read\_data()}} command is used to read data arrays from an ASCII
file.  A typical use would look like
{\small\begin{verbatim}
  Ifeffit> read_data(file = cu_001.xmu, group= 'cu', type = 'xmu')
\end{verbatim}}\noindent
Because of the naming rules for arrays (see \ref{s:datatypes:names}),
{\tt{read\_data()}} needs to assign both a prefix (or group name) and
suffix for each array read in from the data file.  Since data in a file is
usually grouped together logically, {\tt{read\_data()}} will use just one
group name for all the arrays in an individual file.  The group name used
can be specified with the {\tt{group}} keyword in the {\tt{read\_data()}}
command.  By default, the prefix of the filename itself is used.

The method for assigning the {\emph{suffixes}} of the array names is a bit
more involved.  There are four different ways for {\ifeffit} to determine
the array suffixes when reading in a file:
\begin{description}
\item[{\bf{From the 'type' argument}}]: In the example above, the {\tt{type
      = 'xmu'}} argument tells {\ifeffit} to use the suffixes {\tt{energy}}
  and {\tt{xmu}} for the first and second columns, and use {\tt{3}},
  {\tt{4}}, ... for any remaining columns.  Another commonly used
  {\emph{file types}} is {\tt{chi}} for columns of {\tt{k}} and {\tt{chi}}.
  A more complete list of known file types and the suffixes they produce is
  given in the Reference Guide.

\item[{\bf{From the 'label' argument}}]: For data that is not in one of the
  pre-defined types (or if you just want to specify the array suffixes
  explicitly), then the {\tt{label}} argument can be used.  {\tt{label}}
  takes a string that is just the array suffixes listed separated by a
  space. Using {\tt{ read\_data(file = cu\_001.xmu, group='cu', label =
      'energy xmu')}} 
  would be equivalent to the above {\tt{type}} version.

  \item[{\bf{From the files own 'label' line}}]: Many files (especially,
    those written by {\ifeffit}, {\feffit}, or {\autobk}) will have a
    {\emph{label line}} which contains the column labels (i.e,
    array suffixes) and appears just before the data and just after
    a line of minus signs ({\tt{\#-------------}}):
    {\small{
%\begin{latexonly}
 \begin{Sbox}\begin{minipage}{5.00truein}
%\end{latexonly}
\begin{Verbatim}
 # Cu foil at 10K
 # OFFSETS     51284   50016   48319
 #----------------------------------
 #     energy          xmu
  .8786204E+04    .1013661E+01
\end{Verbatim}
%\begin{latexonly}
 \end{minipage}
 \end{Sbox}\setlength{\fboxsep}{2mm}{%
 \begin{flushright}\shadowbox{\TheSbox}\end{flushright}}
%\begin{latexonly}
}}\noindent
    If neither the {\tt{type}} nor {\tt{label}} keyword are specified,
    {\ifeffit}  will look for a label line and use it.
    
  \item[{\bf{By column index}}]: If none of the above methods are used
    (that is neither the {\tt{type}} nor {\tt{label}} keyword are given,
    and a label line is not found), {\ifeffit} will name the arrays by
    column number {\tt{1}}, {\tt{2}}, {\tt{3}}, \ldots.  Though primitive,
    this is actually the most predictable behavior, and can be enforced by
    using "{\tt{type = 'raw'}}".

\end{description}

%%%%%%%%%%%%%%%%%%%%%%%%%%%%%%%%%%%%%%%%%%%%%%%%%%%%%%%%%%%
%%%
\clearpage
\section{Plotting Data}\label{s:plotting}
%%%%%%%%%%%%%%%%%%%%%%%%%%%%%%%%%%%%%%%%%%%%%%%%%%%%%%%%%%%

This section assumes that {\ifeffit} has been built with the PGPLOT
plotting package.  This is available for both Unix and Win32 systems,
though support for Win32 is still experimental.  For Unix systems, this
library has to be installed prior to installing {\ifeffit}, which is fairly
easy to do for most systems.  {\ifeffit} can also be built without this
plotting package.

Plotting in {\ifeffit} is encapsulated in a handful of commands, the most
important of which is the {\tt{plot()}} command, which takes two required
arguments for the ordinate (x-array) and abscissa (y-array) to be plotted,
and take a large set of optional arguments.  A simple plot can be done like
this:
{\small\begin{verbatim}
  Ifeffit> plot(cu.energy, cu.xmu)
\end{verbatim}}\noindent
The {\tt{plot}} command will overplot , so that a second {\tt{plot}}
command:
{\small\begin{verbatim}
  Ifeffit> plot(cu.energy, cu.bkg)
\end{verbatim}}\noindent
will add a trace of the background to the earlier plot.  To force the
current plot to be erase before plotting, you'd say 
{\small\begin{verbatim}
  Ifeffit> newplot(cu.energy, cu.xmu)
\end{verbatim}}\noindent
Each x-y trace plotted has a {\emph{color}} and {\emph{line style}}
associated with it.  You can set the values in these table explicitly for
each particular plot command by specifying the color or linestyle (or both)
directly:
{\small\begin{verbatim}
  Ifeffit> plot(cu.energy, cu.xmu, color = blue)
  Ifeffit> plot(cu.energy, cu.bkg, color = red, style = dashed)
\end{verbatim}
}\noindent
You can also pre-define the color and linestyle for the first, second,
\dots trace ahead of time 
{\small\begin{verbatim}
  Ifeffit> color(1=blue, 2 = red, 3 = black)
  Ifeffit> linestyle(1=solid, 2=dashed, 3 = linespoints2)
\end{verbatim}
}\noindent
so that the first trace plotted is a solid blue line, and the second is a
dashed red line, and the third a black line with a '+' at each data point.

The allowed color names are the 'standard X Windows' colors found in the
{\file{rgb.txt}} file on your system.  Most common color names are
supported, as well as the descriptive if ambiguous 'X Windows' names like
"lightsalmon3" and "bisque".  You may also use the conventional hexadecimal
representation of the color with a string of '\#RRGGBB'.

Allowed line styles are 'solid', 'dashed', 'dotted', 'points', and
'linespointsN' where N = 1, 2, 3, \ldots.  The latter will draw a line
connecting symbols at each data point, with symbols '.', '+', '*','o', 'X',
squares, and triangles for N=1,2,3,4,5, and 6.  Setting the color and
linestyle tables like this is often a convenient thing to do in a macro
(see section~\ref{s:macros}) or start-up file.

For Unix using X Windows, the PGPLOT window supports getting the ($x$,$y$)
coordinates from the plot window using the mouse device.  To use this
within {\ifeffit}, you'd type {\tt{cursor}} at the command prompt, and then
click on the desired point on the plotting window.  The {\ifeffit}
variables {\tt{cursor\_x}} and {\tt{cursor\_y}} will contain the ($x$,$y$)
coordinates.  The {\tt{zoom}} command will let you view a selected region
of the plot by clicking the mouse on the corners of the area to zoom in on.
Many more plotting options exist -- please consult the Reference Guide.

%%%%%%%%%%%%%%%%%%%%%%%%%%%%%%%%%%%%%%%%%%%%%%%%%%%%%%%%%%%
%%%
\clearpage
\section{XAFS Data Processing}\label{s:xafs-process}
%%%%%%%%%%%%%%%%%%%%%%%%%%%%%%%%%%%%%%%%%%%%%%%%%%%%%%%%%%%

{\ifeffit}'s main job is to help you analyze XAFS data, and {\ifeffit}
tries to make it easy to do simple XAFS analysis tasks.  These simple tasks
include such things as converting beamline data in {\muE}, doing pre-edge
subtraction, determining {\Eo}, fitting a post-edge (spline) subtraction to
determine {\bkg} and {\chik}, and doing XAFS Fourier transforms to look at
data in $R$-space.  These standard tasks of XAFS data manipulation are
described in this section.

\subsection{Data Reduction}\label{s:xafs-process:reduce}

The general algebraic manipulation of array data within {\ifeffit} makes
the conversion of raw beamline data to XAFS {\muE} very easy, and also
allows the averaging and re-scaling (if necessary) of data. For example,
reading raw beamline transmission and fluorescence data and converting this
to {\muE} might look like this:

{\small\begin{verbatim}
  Ifeffit> read_data(file = cu_expt.dat, group= 'cu',
                    label = 'energy i0 i1 if')
  Ifeffit> set cu.xmu_t = log(cu.i1 / cu.i0)
  Ifeffit> set cu.xmu_f = (cu.if / cu.i0)
\end{verbatim}}\noindent
If you've collected data in fluorescence using a multi-element-detector,
you may need to sum several arrays before dividing by $I_0$, something
like this will do the trick:
{\small\begin{verbatim}
  Ifeffit> read_data(file= med_fluor.dat, group='med', type='raw')
  Ifeffit> set med.if= (med.4  + med.5 + med.6 + med.7 + med.8 +
                        med.9  +med.10 +med.11 +med.12 +med.13 )
  Ifeffit> set med.xmu= (med.if / med.2)
\end{verbatim}}\noindent


\subsection{Pre-Edge Subtraction, E0 determination, and Normalization}
\label{s:xafs-process:preedge}

Pre-edge subtraction removes the baseline from the EXAFS {\muE},
determines the edge energy (used as the origin of {\emph{k}}), and
normalizes the above-edge {\muE} to 1.  All of this is done by
the {\tt{pre\_edge()}} command which takes arrays of energy and absorption
{\muE}:
{\small\begin{verbatim}
  Ifeffit> pre_edge(cu.energy, cu.xmu)
\end{verbatim}
}\noindent
Like most {\ifeffit} command, this simple-looking command actually causes a
fair amount of data processing behind the scenes.  It also creates several
program variables, especially arrays and scalars detailing the pre-edge
subtraction.  Here's a list of the most important program
variables that {\tt{pre\_edge()}} sets:
\relax\par\smallskip
\begin{tabular}{ll}
 Variable & Description\\
 {\tt{e0}}      & Energy Origin (found near maximum derivative
  of {\muE}\\
 {\tt{edge\_step}} & Edge Step / normalization constant\\
 {\tt{pre\_slope}} & slope of pre-edge line\\
 {\tt{pre\_offset}} & offset of pre-edge line\\
 {\tt{\$group.pre}} & array of pre-edge subtracted {\muE}\\
 {\tt{\$group.norm}} & array of pre-edge subtracted and normalized {\muE}\\
\end{tabular}
\relax\par\smallskip\noindent
where {\tt{\$group}} is the 'group name' taken from the input {\muE} array -
'cu' in this case.  Using {\tt{pre\_edge()}} on another array would
generate new arrays of pre-edge subtracted and normalized {\muE} for the
corresponding group.  Note, however, that it will overwrite the scalar
values from the earlier invocation of {\tt{pre\_edge()}}.

There are several optional arguments to {\tt{pre\_edge()}} not mentioned
here.  These arguments can be use to set the ranges over which to fit the
pre-edge line and post-edge curve, whether or not to perform every part of
the calculation, and so forth.  These optional arguments would normally
be used like this:
{\small\begin{verbatim}
  Ifeffit> pre_edge(cu.energy, cu.xmu, e0=8980.5)
\end{verbatim}}\noindent \noindent
which would force the value of $E_0$ to be $8980.5$ and prevent
{\tt{pre\_edge()}} from trying to determine $E_0$ by itself.
As for all commands, the complete set of the 
optional parameters for 
{\tt{pre\_edge()}} are given in the Reference Guide.

\subsection{Post-Edge Background Subtraction}\label{s:xafs-process:spline}

Post-edge background subtraction involves drawing a ``smooth background''
({\bkg} through the oscillatory part of the XAFS and extracting
{\chik} using this and the formula

\[
\chi(E) = {{\mu(E) - \mu_0(E)}\over{\Delta \mu(E_0)}}
\]
\noindent
where $ \mu_0(E)$ is ``the smooth background'' of {\muE} and $\Delta
\mu(E_0)$ is the edge-jump.  Determining the XAFS background function
{\bkg} generally receives quite a bit of attention in the XAFS community.
{\ifeffit} uses the {\autobk} algorithm, which simply asserts and then
implements a rather common-sense approach to background subtraction: only
the low-frequency components of {\muE} should make up {\bkg}.

The implementation of {\autobk} in {\ifeffit} is encompassed in the
{\tt{spline()}} command, which takes arrays of energy and $\mu$, and several
optional parameters, and writes out {\chik}, {\bkg}, and several
scalars. A basic use of {\tt{spline()}} would look like this
{\small\begin{verbatim}
  Ifeffit> spline(cu.energy, cu.xmu, rbkg=1.0)
\end{verbatim}
}\noindent
Like {\tt{pre\_edge()}}, this simple-looking command does quite a bit of
data processing behind the scenes, and creates or writes several variables
with {\ifeffit}.  For one thing, {\tt{spline()}} will execute
{\tt{pre\_edge()}} unless it's obvious that it shouldn't{\footnote{where
    ``obvious'' means that an array named {\$group.pre} already exists.
    This really is mediocre definition of obvious, especially if
    you're writing scripts to automate the processing of lots of data.  On
    the other hand, if you're casually strolling through your data at a
    command-line prompt, it's probably fine.}}.  That means that all the
output parameters of {\tt{pre\_edge()}} will be created (or overwritten) by
{\tt{spline()}}, and $E_0$ and the edge jump $\Delta \mu(E_0)$ will be
determined if needed.

To say much more about the {\tt{spline()}} command, I'd have to discuss the
details of the {\autobk} algorithm. I'll try to be brief: {\tt{spline()}}
chooses a smooth background spline such that the low-$R$ portion of the
resulting EXAFS $\chi$ are minimized.  That means that {\tt{spline()}}
needs to do a Fourier transform of the $\chik$ it generates.  Because of
this, the {\tt{spline()}} function takes many command arguments that
resemble Fourier transform parameters, so that in addition to the
arguments of {\tt{pre\_edge()}} the important arguments that
{\tt{spline()}} takes are:
\relax\par\smallskip
\begin{tabular}{ll}
 Variable & Description\\
 {\tt{rbkg}}    & $R_{\rm bkg}$, the highest $R$ value to consider
 background\\
 {\tt{kmin}}    & $k_{\rm min}$, the starting $k$ for the Fourier
 transform\\
 {\tt{kweight}} & $w$, the $k$-weight factor for the Fourier transform.
\end{tabular}
\relax\par\smallskip\noindent
Please note that the Fourier transforms appropriate for {\tt{spline()}} are
{\emph{not}} the same as those appropriate for structural analysis.
Usually, values of $k_{\rm min} \approx 0$ and $w = 1$ are
appropriate for {\tt{spline()}}.

Like {\tt{pre\_edge()}}, the {\tt{spline()}} command also creates new
arrays, most importantly:
\relax\par\smallskip
\begin{tabular}{ll}
 Variable & Description\\
 {\tt{\$group.bkg}}    & {\bkg}, the background function itself\\
 {\tt{\$group.k}}      & $k$, the array of wavenumbers\\
 {\tt{\$group.chi}}    & {\chik}, the (unweighted) EXAFS
\end{tabular}
\relax\par\smallskip\noindent

\subsection{Fourier Transforms}\label{s:xafs-process:fft}

Fourier transforms are an integral part of XAFS analysis, and the ability
to perform them quickly and easily is very important.  {\ifeffit} has
different commands for forward ($k\rightarrow R$) and reverse forward
($R\rightarrow q$: I'll use $q$ to refer to back-transformed $k$-space
data) Fourier transforms.  They're very similar to one another, so I'll
first discuss {\emph{forward}} Fourier transforms with {\tt{fftf()}} in
detail, then {\emph{reverse}} Fourier transforms with {\tt{fftr()}} more
quickly.

{\ifeffit} use a simple Fast Fourier transform (FFT), which places some
demands on the data transformed by these commands.  In addition, XAFS
analysis usually imposes some conventions on Fourier transforms, so that
the commands discussed here are not general purpose Fourier transform
functions, but are really {\emph{XAFS Fourier transforms}}.  The most
obvious difference between 'normal' and 'XAFS' Fourier transforms is that
the latter transforms $k$ to $R$, while a 'normal' FT would transform $k$
to $2R$.  Aside from a scale factor, this changes the normalization
constants used. In addition, the XAFS Fourier transform (at least as
{\ifeffit} implements it) allows a variety of smoothing window functions,
and a weighting factor.  More details can be found in {\XAIBook}.

\subsubsection{Forward Fourier Transforms}\label{s:xafs-process:fftf}

The Forward XAFS Fourier transform command is {\tt{fftf()}}.  It's primary
input is an array of {\chik} data.  The requirements of the FFT mean that
the input {\chik} data {\bf{must}} be an array that is on an even $k$-grid
with grid spacing of $k = 0.05 \rm\, \AA^{-1}$, and starting at $k=0$.

These are stringent requirements.  Fortunately, the {\tt{spline()}} command
of section~\ref{s:xafs-process:spline} writes its output {\chik} array
according to these rules.  If you're importing {\chik} data written from
another program, you'll have to make sure the data is moved to this
$k$-grid.  There are two ways to this: either interpolate the data yourself
(see the interpolation functions in the Reference Guide) or specify the
$k$-array corresponding to your $\chi$ data in the {\tt{fftf()}} command
and let it do the interpolation for you.

Properly aligned {\chik} data can be transformed to $R$-space using a
command like this:
{\small\begin{verbatim}
  Ifeffit> fftf(cu.chi, kmin=2.0, kmax=17.0, dk=1.0, kweight=2)
\end{verbatim}}\noindent
while data not on the expected grid would be Fourier transformed like this:
{\small\begin{verbatim}
  Ifeffit> fftf(cu.chi, k=cu.k,
                kmin=2.0, kmax=17.0, dk=1.0, kweight=2)
\end{verbatim}
}\noindent
The keywords {\tt{kmin}}, {\tt{kmax}}, and {\tt{dk}} help define the window
function, and {\tt{kweight}} sets the $k$-weighting factor.  You can also
specify the form of the window function.  The full list of window types and
their functional form is given in the Reference Guide, but the most useful
ones are the Hanning window ( {\tt{kwindow=hanning}} -- the default
window function) which ramps up to one on either end of the $k$-range as
$\cos^2$, and the Kaiser-Bessel window ({\tt{kwindow=kaiser}}), which is
often thought to give superior peak resolution.

Because it is often necessary to do many Fourier transforms with exactly
the same parameters, the {\tt{fftf()}} command, can also read Fourier
transform parameters from appropriately named program variables.  This is
actually a feature of many commands, but it seems most useful for the
Fourier transform commands.  It works like this: setting scalars
{\tt{kmin}},  {\tt{kmax}}, and so forth will have the same effect as
setting those arguments to the {\tt{fftf()}} command. 
{\small\begin{verbatim}
  Ifeffit> kmin=2.0, kmax=17.0, dk=1.0, kweight=2
  Ifeffit> fftf(cu.chi)
\end{verbatim}
}\noindent
would have the result as
{\small\begin{verbatim}
  Ifeffit> fftf(cu.chi, kmin=2.0, kmax=17.0, dk=1.0, kweight=2)
\end{verbatim}
}\noindent
In fact, the {\tt{fftf()}} command will first read the value for the
parameter $k_{\rm min}$ from the program variable {\tt{kmin}} (and so on
for the other Fourier transform parameters), and then from the command
argument {\tt{kmin}}, so that the command argument will always override the
program variable value.  Furthermore, {\tt{fftf()}} will itself set the
value of the program variable {\tt{kmin}}, possibly overwriting any value
you had previously set.   Thus
{\small\begin{verbatim}
  Ifeffit> kmin=2.0, kmax=17.0, dk=1.0, kweight=2
  Ifeffit> fftf(cu.chi, kmin=3.)
  Ifeffit> fftf(other.chi)
\end{verbatim}
}\noindent
will use the same Fourier transform parameters (that is $k_{\rm min} =
3\rm\AA^{-1}$) for both transforms.

A Fourier transform inherently deals with complex data, and a minor
complication arises from the inconvenience that the measured XAFS {\chik}
is strictly a real function.  In {\tt{fftf()}} then, there is an ambiguity
of whether to use the measured XAFS as the real or imaginary part of the
complex XAFS function.  Since the measured XAFS is typically described as
the imaginary part of a complex fine-structure function $\tilde\chi$, one
might be tempted to say that the data {\chik} ought to be set to the
imaginary part of $\tilde\chi$.  {\ifeffit} usually assumes that the data
{\chik} is the real part of the complex $\tilde\chi$ and sets the imaginary
part to zero -- this is in keeping with the convention of older programs
from the University of Washington, but it is purely a matter of convention.
You can explicitly specify the behavior by saying {\tt {fftf(imag =
    data.chi,\ldots)}} or {\tt {fftf(real = data.chi,\ldots)}}.  If you
don't specify, the data will be taken as the real part.  This is almost
never important -- at least not until you want to compare unfiltered
{\chik} with filtered {\chik}.  The output arrays generated by the
{\tt{fftf()}} command are
\relax\par\smallskip
\begin{tabular}{ll}
 Variable & Description\\
 {\tt{\$group.win}}        & The $k$-space window function used \\
 {\tt{\$group.r}}          & $R$, the array of distances\\
 {\tt{\$group.chir\_mag}}  & $|\chi(R)|$, the magnitude of  $\chi(R)$\\
 {\tt{\$group.chir\_pha}}  & the phase of $\chi(R)$\\
 {\tt{\$group.chir\_re}}   & $\rm Re[\chi(R)]$, the real part of  $\chi(R)$\\
 {\tt{\$group.chir\_im}}   & $\rm Im[\chi(R)]$, the imaginary part of  $\chi(R)$
\end{tabular}
\relax\par\smallskip\noindent

It is possible to do ``phase-corrected'' Fourier transforms with {\ifeffit}
using the theoretical phases from {\feff} calculations, but that is beyond
the scope of this tutorial.

\subsubsection{Reverse Fourier Transforms}\label{s:xafs-process:fftr}

{\tt{fftr()}} is the Reverse XAFS Fourier transform command, mainly used to
filter $\chi(R)$ data to backtransformed {\chik}.  Following the
convention of {\feffit}, backtransformed $k$-space is called $q$ to avoid
confusion with the original $k$-space data.  The {\tt{fftr()}} command then
transforms $\chi(R)$ to $\chi(q)$.  As with the forward Fourier transform,
the data is requirements of the FFT mean that the input {\chik} data
{\emph{must}} be given as an array that is evenly spaced in $R$ with a
fixed grid spacing of $R = \pi/1024 \approx 0.03068 \rm\, \AA$, and
starting at $R = 0$.  To further complicate matters, there is
ambiguity as to whether to transform just the real part, just the imaginary
part of $\chi(R)$, or both.  In general, both parts are transformed to
ensure that the overall amplitude scale is preserved.


{\tt{fftr()}} has an almost identical command set to {\tt{fftf()}}, with
{\tt{k}} replaced by {\tt{r}} in the parameter names.  That is, the FT window parameters are
defined with {\tt{rmin}}, {\tt{rmax}}, {\tt{dr}}, and so on.  The window
functional form is set by {\tt{rwindow}}.  A typical use might look like
this
{\small\begin{verbatim}
  Ifeffit> fftr(real=cu.chir_re, imag=cu.chir_im,
                rmin=1.70, rmax=3.0, dk=0.1)
\end{verbatim}}\noindent

The output arrays generated by the {\tt{fftf()}} command are
\relax\par\smallskip
\begin{tabular}{ll}
 Variable & Description\\
 {\tt{\$group.rwin}}       & The $R$-space window function used \\
 {\tt{\$group.q}}          & $q$, the array of wavenumbers\\
 {\tt{\$group.chiq\_mag}}  & $|\chi(q)|$, the magnitude of  $\chi(q)$\\
 {\tt{\$group.chiq\_pha}}  & the phase of $\chi(q)$\\
 {\tt{\$group.chiq\_re}}   & $\rm Re[\chi(q)]$, the real part of  $\chi(q)$\\
 {\tt{\$group.chiq\_im}}   & $\rm Im[\chi(q)]$, the imaginary part of  $\chi(q)$
\end{tabular}
\relax\par\smallskip\noindent
Before ending this section, let me say a few words about ``Fourier
filtering''.  {\ifeffit}'s approach to Fourier transforms is fairly
general, and the use of two Fourier transforms to ``isolate a shell'' is
not a trivial process with {\ifeffit}.  This partly reflects my experience
and belief that ``Fourier filtering'' can {\emph{not}} be made a trivial
process.

Comparing filtered with unfiltered data is a common desire. Given the
ambiguities in how to handle the real and imaginary parts of the data,
getting the details right for this are not trivial.  Though macros won't be
discussed until section~\ref{s:macros}, using a macro to get the details
right is a good idea.  Such a macro for Fourier filtering might look like
this:
{\small\begin{verbatim}
  macro filter group  "kweight=2,kmin=3"   "dr=0"
    fftf(real=$1.chi, $2)
    fftr(real=$1.chir_re, imag=$1.chir_im,  $3)
    set $1.chik   = $1.chi  * $1.k^kweight
    set $1.chik_w = $1.chik * $1.win
    set $1.chiq   = $1.chiq_re / ($1.k^kweight)
  end macro
\end{verbatim}}\noindent
With this macro definition (which takes arguments as group name, $k$
parameters, and $R$ parameters), you could perform a filter, and overplot
filtered and unfiltered data with
{\small\begin{verbatim}
  filter data "kweight=2,kmin=3,kmax=15,dk=1" "rmin=1.6,rmax=3."
  newplot(data.q, data.chiq_re )
  plot(   data.k, data.chik_w )
\end{verbatim}}\noindent
Note that the convention used here is that the data {\chik} is the real
part for the Forward transform, so that the real part of $\chi(q)$ is
the appropriate choice for comparison.  Of course, that should be
compared to the $k$-weighted, windowed {\chik}.


%%%%%%%%%%%%%%%%%%%%%%%%%%%%%%%%%%%%%%%%%%%%%%%%%%%%%%%%%%%
%%%
\clearpage
\section{XAFS Analysis}\label{s:xafs-anal}
%%%%%%%%%%%%%%%%%%%%%%%%%%%%%%%%%%%%%%%%%%%%%%%%%%%%%%%%%%%

XAFS Analysis in {\ifeffit} consists of modeling XAFS {\chik} in terms of a
{\emph{sum of paths}} with the XAFS contribution from each path being
modeled by a calculation from {\feff}.  A Path is an abstract formalism for
breaking the EXAFS into manageable pieces based on the scattering path that
a photo-electron takes.  Many EXAFS analysis methodologies use concepts
such as 'shell' or 'coordination sphere'.  To some extent the distinction
is only conceptual, and therefore unimportant.  In other ways, the path
formalism is clearly a superior way to robustly analyze complex EXAFS data
that includes the effects of multiple scattering.  If you're used
to thinking of your EXAFS data as having contributions based on shells or
coordination spheres, paths aren't that big of a change.

\subsection{Defining Paths}\label{s:xafs-process:path}

{\ifeffit} relies on {\feff} for basic information about all of its paths.
This means that you need to run {\feff} and sort through its results
(neither of which are trivial tasks) prior to defining paths with
{\ifeffit}.  The outputs of {\feff} that {\ifeffit} needs is the set of
{\feffndat} files, where each file represents the result for a single path.
Some versions of {\feff} can write all of the path information into a
single file called {\feffbin} as an alternative to a large set of
{\feffndat} files.  {\ifeffit} can read some versions of {\feffbin}, but
this discussion will focus on using the more universal {\feffndat} files.

Once {\feff} has run and a set of {\feffndat} files exists, using them
within {\ifeffit} is fairly easy.  The {\tt{path()}} command is used to
define paths within {\ifeffit}.  The information needed to describe an XAFS
path is a bit too complicated to conveniently hold as a set of {\ifeffit}
scalars and arrays -- it could be done, but the bookkeeping would be a
nightmare.  Instead, paths are defined and stored internally, so that you
can refer to a path by an integer index.  Though a break from the idea in
section~\ref{s:datatypes} that all data is stored in program variables that
you can access easily and directly, there are commands to convert the path
data to program variables.

The {\tt{path()}} command takes the name of a {\feffndat} file and a unique
{\ifeffit} {\emph{path index}} as its primary arguments.  The {\ifeffit}
path index is an integer (from 1 to 10000).  It does not need to be the
same index as {\feff} used, but that is often a convenient way to do it.
In addition, the {\tt{path()}} command takes several optional arguments,
known as {\emph{Path Parameters}\/} that represent the parameters used to
modify the EXAFS function for that path.  The path parameters include such
things as $\Delta R$, the change in path distance (well, half-path-length
for multiple scattering paths), $\sigma^2$, the mean-square displacement of
the bond, and an $E_0$ shift.  The most important Path Parameters are

\begin{tabular}{ll}
 Path Parameter keyword & Description\\
 {\tt{file}}      & Name of {\feffndat} file\\
 {\tt{index}}     & path index: a unique integer identification\\
 {\tt{label}}     & path label: a string describing the path \\
 {\tt{s02}}       & $S_0^2$, amplitude reduction factor\\
 {\tt{e0}}        & $E_0$, energy shift\\
 {\tt{delr}}      & $\Delta R$\\
 {\tt{sigma2}}    & $\sigma^2$\\
 {\tt{third}}     & $C_3$, third cumulant\\
\end{tabular}
\relax\par\smallskip\noindent
a full list is given in the Reference Guide.
A simple path definition would look like this:

{\small\begin{verbatim}
  Ifeffit> path(index=1,file = feff0001.dat,
                s02 = 0.9, sigma2 = 0.001)
\end{verbatim}
}\noindent 
This would define path 1, and assign values for $S_0^2$ and $\sigma^2$ for
this path.  An important aspect of {\ifeffit} is that the Path Parameters
are {\emph{defined}} in the same way as other {\tt{define}}d scalars (see
section~\ref{s:datatypes}) -- the formulas specified for the path
parameters in the {\tt{path()}} command are remembered.  This allows you to  
define the path like this

{\small\begin{verbatim}
  Ifeffit> s02_001  = 0.9
  Ifeffit> e0_001   = -1.
  Ifeffit> sig2_001 = 0.001
  Ifeffit> path(index=1,file = feff0001.dat,
                s02    = s02_001,
                e0     = e0_001,
                sigma2 = sig2_001 )
\end{verbatim}}\noindent
where the parameter {\tt{s02}} for path 1 now depends on the definition of
the scalar {\tt{s02\_001}}.  This is a fairly simple example, and the
definition for the {\tt{s02}} parameter for path 1 could be more
complex -- say, {\tt{min(1.2,s02\_001)}} to set an upper bound on this
parameter.   In addition, the {\tt{s02\_001}} scalar here could have a more
complex definition or even be a fitting variable defined with {\tt{guess
    s02\_001 = 0.9}}.

Additional paths are defined in the same way:
{\small\begin{verbatim}
  Ifeffit> path(index=2,file = feff0002.dat,
                s02    = s02_001, e0 = e0_001,
                sigma2 = sig2_001 )
\end{verbatim}}\noindent
The only {\emph{required}} keyword for a path is the integer index (if the
keyword comes first in the list of arguments, the {\tt{index}} can be
dropped, so a completely equivalent definition is
{\small\begin{verbatim}
  Ifeffit> path(2, file   = feff0002.dat)
  Ifeffit> path(2, s02    = s02_001  )
  Ifeffit> path(2, e0     = 2.0      )
  Ifeffit> path(2, sigma2 = sig2_001 )
\end{verbatim}}\noindent
subsequent {\tt{path()}} commands will set or {\emph{overwrite}} the
definitions for other path parameters.  This makes it fairly simple to
change the definition of a path parameter half way through an analysis, say
between fits, where something like
{\small\begin{verbatim}
  Ifeffit> path(2, s02    = 0.85)
\end{verbatim}}\noindent
will change one path parameter and leave the rest as previously defined.

\subsection{Combining Paths}\label{s:xafs-process:ff2chi}

Once defined, combining paths to give a {\chik} function is very easy with
the command {\tt{ff2chi()}}.  {\tt{ff2chi()}} takes its name from the {\feff}
module that combine {\feff} path files into a single {\chik} function.
Because {\ifeffit} allows you to alter the paths through the Path
Parameters, and use paths calculated from different runs of {\feff}, this
version of {\tt{ff2chi()}} is significantly more flexible and powerful than
{\feff}'s own version.

{\tt{ff2chi()}} takes a list of path indices, and creates arrays for
{\chik}.  Assuming that the paths have already been defined, a command
like this:
{\small\begin{verbatim}
  Ifeffit> ff2chi(1,2,3, group = ff)
\end{verbatim}}\noindent
\noindent
will add paths 1, 2, and 3 together and create arrays {\tt{ff.k}} and
{\tt{ff.chi}} containing the sum of these paths.  Each of the paths will be
altered according to its own Path Parameters in the sum.  The list syntax
'{\tt{1, 2, 3}}' could also be replaced by '{\tt{1-3}}'.  Lists of the form
{\tt{1-3,5,7-10,23,11}} are also allowed.  An important note is that
(currently), paths listed twice are used twice.  This is an easy mistake to
make, so beware{\footnote{it is on the ``TODO list'' to check for and warn
    about this possibility, which is hardly ever intended}}.

Besides the list of paths and group name for the output arrays,
{\tt{ff2chi()}} has a few other arguments, most of which you won't really
need.  I'll just mention a few convenient ones here. The arguments
{\tt{kmin}} and {\tt{kmax}} can set the $k$-range of the output arrays, and
the argument {\tt{do\_real}} will cause the ``real part'' of {\chik} to be
written to the array {\tt{\$group.chi\_real}} -- the confusion over real
and imaginary parts of the complex {\chik} shows up again!  The
{\tt{\$group.chi}} array that {\tt{ff2chi()}} always generates is the
imaginary part of the complex {\chik}, and should correspond to the
experimentally derived {\chik}.


\subsection{Getting and Viewing Path Parameters}\label{s:xafs-process:get_path}

Once you've combined a few paths with {\tt{ff2chi()}} or, as we'll see
shortly, done a fit with {\tt{feffit()}}, you may want to find out some
information about the path parameters for a set of paths.  There are a few
ways to get this information.  The {\tt{show()}} command, first discussed
in section~\ref{s:datatypes:show}, has a ``{\tt{@path}}'' modifier to give
some information about a defined path.
{\small\begin{verbatim}
  Ifeffit> show @path=1
   PATH   1
     feff  = feffcu01.dat
     id    = Cu metal first neighbor
     reff  =    2.547800, degen  =   12.000000
     s02   =    0.937373, e0     =    0.510790
     dr    =    0.000525, ss2    =    0.003496
     3rd   =    0.000000, 4th    =    0.000000
     ei    =    0.000000, dphase =    0.000000
\end{verbatim}}\noindent
This shows all the scalar values of the path parameters ({\tt{reff}} is
the half-path length, {\tt{degen}} is the path degeneracy, and so on).
The argument to the {\tt{@path}} modifier to {\tt{show()}}  can be any
valid path list as described in section~\ref{s:xafs-process:ff2chi}
(for example: {\tt{1,2,4-9,12}}).  To see all the defined paths
{\tt{show @paths}} will work.

The {\tt{show @path}} family of commands only prints the values of the path
parameters.  Another way to get path information is to convert them to
program variables with the {\tt{get\_path()}} command.  This command only
takes one path at a time (not a path list), but allows you to get the path
parameter values into program variables for later manipulation.  The syntax
for {\tt{get\_path()}} is
{\small\begin{verbatim}
  Ifeffit> get_path(path = 1, prefix = path1)
\end{verbatim}
}\noindent
or, equivalently and more simply
{\small\begin{verbatim}
  Ifeffit> get_path(1, path1)
\end{verbatim}
}\noindent
This will create scalars {\tt{path1\_s02}}, {\tt{path1\_e0}},
{\tt{path1\_ei}}, {\tt{path1\_delr}}, {\tt{path1\_sigma2}},
{\tt{path1\_third}}, {\tt{path1\_fourth}}, {\tt{path1\_degen}},
{\tt{path1\_reff}} and give the values of the path parameters for path 1.
In addition, the strings {\tt{\$path1\_file}} and {\tt{\$path1\_id}} will
contain the values of the path {\feff} file name and path identification
string, respectively.  The {\tt{prefix}} argument to
{\tt{get\_path()}} can be any valid variable name.
Of course, array variables for a given path can be formed with the
{\tt{ff2chi()}} command, as described in the previous section.  Just to be
clear, though
{\small\begin{verbatim}
  Ifeffit> ff2chi(1, group = path1)
\end{verbatim}}\noindent
\noindent
will create arrays of {\tt{path1.k}} and {\tt{path1.chi}}.  Other arrays
for the real part, magnitude, and phase of {\chik} can be obtained from
{\tt{ff2chi()}} by using arguments {\tt{do\_real}}, {\tt{do\_mag}}, and
{\tt{do\_phase}}, respectively.


\subsection{{\tt{feffit()}}: Fitting XAFS Data with
  Paths}\label{s:xafs-process:feffit}

At some point in the analysis of XAFS data, you'll want to refine a set of
path parameters so that a sum of paths best matches the data.  The
{\tt{feffit()}} command is the basic tool for fitting EXAFS data to a set
of paths.  In some sense, {\tt{feffit()}} is just a fancy version of
{\tt{ff2chi()}}.  Another view might be that the {\tt{feffit()}} command is
the main point of the {\ifeffit} library, so that {\tt{ff2chi()}} is a very
simple version of {\tt{feffit()}}.  In any event, the two functions are
closely related and purposefully implemented with as much overlap as
possible so that once you feel comfortable with the {\tt{paths()}} and
{\tt{ff2chi()}} commands, {\tt{feffit()}} should be a fairly small step.

{\tt{feffit()}} has four basic requirements:
\begin{enumerate}
\item $\chi(k)$ data to fit,
\item a list of paths to sum  to make the model XAFS.
\item Fourier transform and fit range parameters
\item one or more variables defined that affect the model XAFS.
\end{enumerate}
\noindent

First, {\tt{feffit()}} needs some {\chik} data to fit.  As you can probably
guess, this data needs to be a named array for {\chik} just as you would
use for the Fourier transform routines (ie, starting at $k=0$, and
progressing in steps of $0.05\rm\,\AA^{-1}$).  You specify the {\chik}
array with {\tt{feffit(chi= data.chi,\ldots)}}.  If the data you have is
not on the expected grid, you'll either need to interpolate it on to the
expected grid or specify the array of $k$ values, with
{\small\begin{verbatim}
  Ifeffit> feffit(chi= data.chi,k=data.k, ...)
\end{verbatim}
}\noindent
Second, {\tt{feffit()}} needs a list of paths for the sum-over-paths that
makes up the model XAFS.  This sum is essentially the same as for
{\tt{ff2chi()}}, comprising of a simple list of path indices.  Path lists
of the '{\tt{1, 2, 3}}', '{\tt{1-3}}', and '{\tt{1-3,5,7-10,23,51}}' are
accepted, so that the {\tt{feffit()}} command would now look like
(assuming the data is on the proper $k$ grid)
{\small\begin{verbatim}
  Ifeffit> feffit(chi= data.chi,1-3, ...)
\end{verbatim}
}\noindent
Third, {\tt{feffit()}} needs Fourier transform and fit range parameters
defined.  This reflects the fact that XAFS is a band-limited phenomenon,
with a finite $k$- and $R$-range.  As mentioned in
section~\ref{s:xafs-process:fftf}, the Fourier transform parameters can be
read either from the appropriately named program variables ({\tt{kmin}},
{\tt{kmax}}, {\tt{kweight}}, {\tt{dk1}}, {\tt{dk2}}, and so forth) or from
the command argument list (using these same identifiers as keywords).
In addition to the $k$-space Fourier transform parameters, {\tt{feffit()}}
needs the fit $R$-range, specified by {\tt{rmin}} and {\tt{rmax}} 
(this reflects the fact that {\tt{feffit()}} normally does the fit in
$R$-space -- fitting in $k$-space is possible, as well).  Like the Fourier
transform parameters, the $R$-range can be set from program variables
({\tt{rmin}} and {\tt{rmax}} or in the {\tt{feffit()}} command statement itself.
We could say either
{\small\begin{verbatim}
  Ifeffit> kmin = 3.0, kmax = 15.0, kweight = 2, dk1 = 2, dk2= 2
  Ifeffit> rmin = 1.5, rmax =  3.5
  Ifeffit> feffit(chi= data.chi, 1-3 )
\end{verbatim}
}\noindent
or
{\small\begin{verbatim}
  Ifeffit> feffit(chi= data.chi,   1-3, rmin=1.5, rmax=3.5, 
                  kmin=3,kmax=15,kweight=2,dk=2)
\end{verbatim}
}\noindent

The fourth requirement for a fit may seem obvious: {\tt{feffit()}} needs
some variables to vary.  {\bf{Fitting Variables}} are like regular scalars,
except that they are defined with the {\tt{guess()}} command instead of
{\tt{set()}}, as in {\tt{ guess e0 = 1.}} .  Fitting variables also carry
around information about their estimated uncertainty, and correlation with
other fitting variables.  We'll discuss that in the next section.

Putting it all together, a simple fit of a single {\feff} path to {\chik}
data would look like this:
{\small\begin{verbatim}
  Ifeffit> read_data(file = cu_chi.dat, group= data, type = chi)
  Ifeffit> set   s02_001  = 0.9
  Ifeffit> guess e0_001   = 1.
  Ifeffit> guess dr_001   = 0.01
  Ifeffit> guess sig2_001 = 0.001
  Ifeffit> path(index  = 1,
                file   = feff0001.dat,
                s02    = s02_001,
                e0     = e0_001,
                delr   = dr_001,
                sigma2 = sig2_001 )
  Ifeffit> kmin = 3.0, kmax = 16.0, kweight = 2, dk1 = 2
  Ifeffit> rmin = 1.5, rmax = 3.2
  Ifeffit> feffit(data.chi, 1, group=fit)
\end{verbatim}
}\noindent
The {\tt{feffit()}} command will update the values of all the fitting
variables, and also create arrays for the best-fit {\chik} and {\chir}
using the group name specified ({\tt{fit}} in this case, {\tt{feffit}} by
default), and also create arrays for the data {\chir}, effectively running
{\small\begin{verbatim}
  Ifeffit> fftf(data.chi)
  Ifeffit> fftf(fit.chi)
\end{verbatim}
}\noindent
for you.  The best-fit values and estimated uncertainties for the variables
can be seen with {\tt{show @variables}}, as will be further discussed in
the next section.

In addition to what has been discussed so far, {\tt{feffit()}} has several
optional parameters for a wide range of things such as specifying
uncertainty in the input data, whether to refine background-spline
parameters with the structural refinement, how to fit in $k$-space (with or
without Fourier filtering), and an {\ifeffit} macro to run for each fit
iteration.  In the interest of brevity, I'll resist discussing any of these
fascinating topics in this brief tutorial.

\subsection{Uncertainties in Fitted Variables, Fitting Statistics}
\label{s:xafs-process:uncertainties}

A fit is not very useful without some idea of the reliability of the fit
and some estimate of the uncertainty in the fitted variables.  The
{\tt{feffit()}} command automatically calculates the estimated
uncertainties and goodness-of-fit parameters for you.  The particulars of
how these values are determined is outside the scope of this tutorial --
the attention here will be on how to view and manipulate the values for
these statistics that {\tt{feffit()}} determines.

The {\tt{feffit()}} command writes several fitting statistics to help you
determine the goodness of fit.  First, {\ifeffit} will write the number of
fit iterations to {\tt{\&fit\_iteration}}.  Though not necessarily useful
as a fitting statistic, this can be valuable to see if a fit is taking far
too many iterations, probably indicating a problem with the set up or
model.  The number of variables is stored in {\tt{n\_varys}}.  The number
of independent points in the band-limited data is estimated as $ { {2
    \Delta k \Delta R}/\pi } $, for fit $k$-range $\Delta k$ and $R$-range
$\Delta R$ and stored in {\tt{n\_idp}}.  The estimated uncertainty in the
data itself is stored in the variables {\tt{epsilon\_r}} for the
uncertainty in the data {\chir} and {\tt{epsilon\_k}} for the uncertainty
in the data {\chik}.


The main goodness-of-fit statistics are {\tt{r\_factor}},
{\tt{chi\_square}}, and {\tt{chi\_reduced}}, which are three different ways
of scaling the sum-of-squares of the final misfit.  The full definition of
these is given elsewhere, but briefly, {\tt{r\_factor}} is the misfit
scaled to the data itself, so it a {\emph{relative misfit}},
{\tt{chi\_square}} is scaled to the estimated uncertainty in the data
({\tt{epsilon\_r}} for data fit in $R$-space).  {\tt{chi\_reduced}} is
{\tt{chi\_square}} divided by the number of ``free parameters in the fit'',
given by the difference of {\tt{n\_idp}} and {\tt{n\_varys}}.  More
information on these terms and concepts is elsewhere in the {\ifeffit}
documentation.

For each fitting variable, {\ifeffit} will create (or overwrite) a variable
named {\tt{delta\_VAR}}, and write the estimated uncertainty for variable
{\tt{VAR}} to this variable.  For example, the fit shown above would put
the uncertainties in the fitting variables in {\tt{delta\_s02\_001}},
{\tt{delta\_e0\_001}}, {\tt{delta\_sig2\_001}}, and {\tt{delta\_dr\_001}}.

The estimated uncertainties are, of course, very valuable for assessing fit
results.  In addition, it often instructive to know the correlations
between pairs of variables resulting from the fit.  To extract the
correlations from {\ifeffit}, you use the {\tt{correl()}} command, which
simply takes the names of two fitting variables, and the name of an output
scalar variable to store the resulting correlation to.  Thus,
{\small\begin{verbatim}
  Ifeffit> correl(s02_001, sig2_001, out= cor1)
  Ifeffit> print cor1
   0.887130839 
\end{verbatim}}\noindent
\noindent
At this point, it is an error to give {\tt{correl()}} the name of any
variable that is not a fitting variable. 

%%%%%%%%%%%%%%%%%%%%%%%%%%%%%%%%%%%%%%%%%%%%%%%%%%%%%%%%%%%
%%%
\clearpage
\section{Modeling non-XAFS data}\label{s:minimize}
%%%%%%%%%%%%%%%%%%%%%%%%%%%%%%%%%%%%%%%%%%%%%%%%%%%%%%%%%%%

In the course of data analysis, it is often necessary to fit non-XAFS data.
This can be any task similar to fitting a line or quadratic function to a
set of data (say $\sigma^2 v. T$), fitting a set of Lorentzian, Gaussian
and/or arc-tangent functions to a XANES spectra, or fitting some weighted
average of model XANES spectra to fit an unknown spectra.  Building on the
general data-processing and XAFS modeling capabilities, {\ifeffit}
provides a simple and powerful way to model many kinds of data.

The {\tt{minimize()}} command implements a simple least-squares
minimization routine.  It takes as its primary argument a {\emph{residual}}
array to minimize in the least-squares sense.  Usually the residual would
be ``Data - Fit'', but you may want to provide some scaling, possibly
emphasizing some region of the fit more than others.

The residual array is built just like any other array, and is expected to
depend through definitions on a few variables defined with {\tt{guess()}}.
A very simple example would be a linear fit:
{\small\begin{verbatim}
  Ifeffit>read_data(my_data.dat, group = 'data', label = 'x y')
  Ifeffit>guess ( m  = 1, b = 0 )
  Ifeffit>fit.y     = m * data.x + b
  Ifeffit>fit.resid = fit.y  - data.y
  Ifeffit>minimize(fit.resid)
\end{verbatim}
}\noindent
This will adjust the variables {\tt{m}} and {\tt{b}} to best-fit the data.
As with {\tt{feffit()}}, the fitting statistics {\tt{\&fit\_iteration}},
{\tt{chi\_square}}, {\tt{chi\_reduced}}, and {\tt{r\_factor}} will be
written and uncertainties in the variables {\tt{delta\_m}} and
{\tt{delta\_b}} will be created for the uncertainties in the fitted
parameters.  The {\tt{correl()}} function described above can be used to
retrieve the correlation between 2 fitting variables. Since the general
problem of estimating uncertainties in experimental data is difficult,
you'll need to be quite mindful of this fact and careful when interpreting
these fitting statistics.

In order to specify an estimate of the uncertainty in the experimental
data, create an array (of the same size as the residual array) containing
the estimated  uncertainty at each point, and tell {\tt{minimize()}} to use
this with the {\tt{uncertainty}} argument:
{\small\begin{verbatim}
  Ifeffit> fit.eps = 0.01 * indarr(npts(fit.resid))
  Ifeffit> minimize(fit.resid, uncertainty = fit.eps)
\end{verbatim}
}\noindent

If you wish to limit the fit to be over a limited range of the residual
array, you need to tell {\tt{minimize()}} what array to use as the
ordinate or ``x''-array, and the minimum and  maximum ``x'' values to use:
{\small\begin{verbatim}
  Ifeffit> minimize(fit.resid, x =data.x, xmin= 2, xmax =10)
\end{verbatim}
}\noindent
There are several built-in special mathematical functions to aid in the
modeling of data -- more information can be found in the Reference Guide.


%%%%%%%%%%%%%%%%%%%%%%%%%%%%%%%%%%%%%%%%%%%%%%%%%%%%%%%%%%%
%%%
\clearpage
\section{Saving data and logging your {\ifeffit} session}\label{s:log-save}
%%%%%%%%%%%%%%%%%%%%%%%%%%%%%%%%%%%%%%%%%%%%%%%%%%%%%%%%%%%

This section is about getting information back from {\ifeffit}, either at
the command-line, or written to external files.

\subsection{Writing output data files}
\label{s:log-save:write-data}

At some point, you'll want to write out some of {\ifeffit}'s arrays to a
file.  This is done fairly simply with the {\tt{write\_data()}} command,
which will write a plain ASCII file containing text strings and scalar
values at the top of the file as ``comment lines'', then arrays in column
format.  The syntax for {\tt{write\_data()}} is fairly simple, consisting
of a file name, and a list of arrays, scalars, and strings to write out, as
in
{\small\begin{verbatim}
  Ifeffit> $title1 = 'Fit #1, varying 4 variables'
  Ifeffit> write_data(file='x_fit.chi', rmin, rmax,
                      fit.k, fit.chi, $title1)
\end{verbatim}
}\noindent
which will write the file {\file{x\_fit.chi}}, which will have header lines
from {\tt{\$title1}}, {\tt{rmin}}, {\tt{rmax}} looking like this:
{\small{
%\begin{latexonly}
 \begin{Sbox}\begin{minipage}{5.00truein}
%\end{latexonly}
\begin{Verbatim}
 # Fit #1, varying 4 variables
 # rmin =    1.5000000
 # rmax =    3.5000000
 #------------------------
 #   k          chi
   0.0000000        0.40189216E-01
\end{Verbatim}
%\begin{latexonly}
 \end{minipage}
 \end{Sbox}\setlength{\fboxsep}{2mm}{%
 \begin{flushright}\shadowbox{\TheSbox}\end{flushright}}
%\begin{latexonly}
}}\noindent
followed by columns for the
arrays {\tt{fit.k}} and {\tt{fit.chi}}.  Notice that the file will have a
label line making it ready to be read back in with the {\tt{read\_data()}}
command using this ``label'' line, as described in  section~\ref{s:fileio}.

The string program variables are always written first, in the order they
appear in the argument list, followed by the scalars (as shown) in the
order they appear in the argument.  Finally, the arrays will be written in
the order they appear, with the first one being in the first column.

\subsection{Writing log files}
\label{s:log-save:log}

While doing a complex analysis, it's often necessary to save values and
information into a file for later inspection.  While not extremely
sophisticated, one simple way to do this is to have everything that would be
written to the screen written to a {\emph{log file}}.  This is surprisingly
effective way of keeping track of your analysis session, and may well
inspire you to insert many more {\tt{show()}} and {\tt{print()}} commands
in your scripts.

To open a log file, you use the {\tt{log()}} command, naming the log file
to use, and specify the ``screen echo'' mode:
{\small\begin{verbatim}
  Ifeffit> log(my_test.log, screen_echo = 3)
\end{verbatim}
}\noindent
where the ``{\tt{screen\_echo = 3}}'' selects writing to both the screen
and to the log file.  Other values for this parameter are discussed in the
Reference Guide.   

To close a log file, you say
{\small\begin{verbatim}
  Ifeffit> log(close)
\end{verbatim}
}\noindent
Only one log file can be open at a time -- opening a log file when one is
already open will cause the first to be closed.  If the program exits with
a log file opened, it is not guaranteed that all information will be
actually written to the log file.  (NB: This is under investigation, and is
hoped to be fixed in a future version).


\subsection{Saving the state of an {\ifeffit} session}
\label{s:log-save:save-restore}

It's often desirable to suspend an analysis session and save the current
point in the session, either to return to it later or to compare your
results with someone else's.  To this end, {\ifeffit} allows you to save
the current state of all program variables into a single file that can be
read in later.  The saved file is portable across different platforms.

The {\tt{save}} command will save the current state.  It takes a few
optional arguments, the most important of which is the file name, which
defaults to {\file{ifeffit.sav}}.  That is
{\small\begin{verbatim}
  Ifeffit>save
\end{verbatim}}\noindent
will write {\file{ifeffit.sav}}, and

{\small\begin{verbatim}
  Ifeffit>save( current.sav)
\end{verbatim}}\noindent
will save the program state to {\file{current.sav}}.  You can also specify
what kinds of program variables to save with optional arguments like
{\tt{with\_strings}}, {\tt{no\_strings}}, {\tt{with\_arrays}},
{\tt{no\_arrays}}, and so forth.  The default is to save everything.

To restore a previously saved session, you use the {\tt{restore}} command,
giving it the name of the save file (again, {\file{ifeffit.sav}\/}  by
default):
{\small\begin{verbatim}
  Ifeffit>restore( my_save_file.sav)
\end{verbatim}}\noindent
This will load all the program variables saved in the given file. Note that
it will {\emph{not}} erase any program variables that may have existed in
the current session, but will overwrite any variables with the same name.
This means that unless you start with a new session, it's possible that
restoring a session may not produce an identical session to the previous
one.

%%%%%%%%%%%%%%%%%%%%%%%%%%%%%%%%%%%%%%%%%%%%%%%%%%%%%%%%%%%
%%%
\clearpage
\section{Defining and Using Macros}\label{s:macros}
%%%%%%%%%%%%%%%%%%%%%%%%%%%%%%%%%%%%%%%%%%%%%%%%%%%%%%%%%%%

Macros are named sequence of {\ifeffit} commands which can be run by simply
typing the macro name.  Macros are primarily designed to cut down on typing
for repetitive tasks and to make {\ifeffit} easier to use and customize.
While not going far enough to making {\ifeffit} a real programming
language, macros offer some flexibility normally associated with batch
processing languages.

A macro is defined with the {\tt{macro}} keyword, after which comes a
series of commands as you would type them to 
the command line and ending with {\tt{end macro}}.  As a simple example, 
defining a macro named {\tt{make\_ps}} would look like this
would be
{\small\begin{verbatim}
 macro make_ps
   plot(device="/ps",file= "ifeffit.ps")
 end macro
\end{verbatim}
}\noindent
Now, typing the new command {\tt{make\_ps}} will create a postscript file
named {\file{ifeffit.ps}} of the current plot.  This macro is only one line
long, which doesn't save much typing, but we'll get to longer examples
below.  First, let's add an optional argument: what if you want the
Postscript file to be named something other than {\file{ifeffit.ps}}, since
multiple executions would just keep overwriting this file?  For this, you
can supply {\emph{macro arguments}}, which are variables named {\tt{\$1}},
{\tt{\$2}},\ldots, {\tt{\$9}} that are expanded in place as text strings by
the first, second, \ldots, ninth arguments when the macro is run.  The
definition look like this
{\small\begin{verbatim}
 macro make_ps
   plot(device="/ps",file= "$1")
 end macro
\end{verbatim}}\noindent  % $2
and would be invoked by {\tt{make\_ps data\_01.ps}} or {\tt{make\_ps
    my\_fit.ps}}.  OK, but now what happens if you {\emph{don't}} give
an argument?  Well, the argument {\tt{\$1}} would be "", so that would
be the same as saying {\tt{\ldots,file="")}}, which would create the
    default file {\file{ifeffit.ps}}.

You can also define a default parameter for the macro at its
definition, by including the default value on the ``{\tt{macro}}'' line.  
That is, 
{\small\begin{verbatim}
 macro make_ps  other.ps
   plot(device="/ps",file= "$1")
 end macro
\end{verbatim}
}\noindent  % $2
the default value for {\$1} will be {\tt{other.ps}}.  The value for this
parameter can still be overridden by saying {\tt{make\_ps data\_01.ps}}.

Macros can be of arbitrary length (though the total number of macro lines
in memory is fixed at 2048 lines).  In addition, if the first line of a
macro definition is a plain text string, this will be used as the macro
documentation, and will be shown with {\tt{show @macros}}. A more 
typical macro may look like this
{\small\begin{verbatim}
 macro do_pre_edge  a
    "Read File, Calculate Pre-Edge, Plot, Write File"
    read_data($1.xmu,  type = xmu,  group = my)
    pre_edge(my.energy, my.xmu)
    my.norm = my.pre / edge_step
    $title1 = 'normalized, pre-edge subtracted data'
    write_data(file = $1.pre, $title1,  energy, pre, norm, xmu)
 end macro
\end{verbatim}}\noindent
More example macros are included in the main {\ifeffit} distribution.

%%%%%%%%%%%%%%%%%%%%%%%%%%%%%%%%%%%%%%%%%%%%%%%%%%%%%%%%%%%
%%%
\clearpage
\section{The {\tt{ifeffit}} command-line program}\label{s:cmdline}
%%%%%%%%%%%%%%%%%%%%%%%%%%%%%%%%%%%%%%%%%%%%%%%%%%%%%%%%%%%

The main program distributed with {\ifeffit} is simply called
{\tt{ifeffit}} (or {\tt{ifeffit.exe}} on Win32 systems) and runs the basic
command interpreter we've been discussing throughout this tutorial.  If
you've been following along with the examples, you've run this program
already.  This, the final section of the tutorial, will give a few
additional details of using this program.  The emphasis here will be on the
Unix version.

On Unix and Unix-like systems, the {\tt{ifeffit}} program uses the GNU
Readline library which supports command-line editing, which greatly
enhances the usability of the command-line program by letting you 1) recall
previous commands, 2) edit them with keyboard commands, and 3) use
tab-completion.  In short, the up-arrow and down-arrow keys work to scroll
through previous commands, you can easily edit the command-line, and you
can use the tab key to help you finish typing commands and file names.

Following the usual GNU Readline behavior, 'emacs keybindings' are used,
and there is a large number of 'control-keyboard-sequences' to assist
editing the current command line.  For example Ctrl-a (that's holding the
control key and 'a' at the same time) will move to the beginning of the
line, Ctrl-e will move to the end of the line.  Both Ctrl-f and the
right-arrow key will move one character to the right, and Ctrl-b and the
left-arrow key will move one character to the left.  Ctrl-p is the same as
the up-arrow key, and Ctrl-n is the same as the down-arrow key, and will
scroll through the previous and next commands in the ``history buffer''.
Ctrl-k will cut the text from the current cursor position to the end of the
line, and Ctrl-y will insert that cut text at the current point of the
cursor.  Following the behavior of many shell programs, Ctrl-d has a dual
purpose: on an empty line it will exit the program (essentially the same as
typing 'quit'!), while on a non-empty line it will erase one character at a
time.  Ctrl-h, on the other hand, will erase one character at a time
backwards.  There are several more control-sequences available for more
rapid command-line editing -- consult the Readline documentation.

Tab completion means that if you've typed {\tt{read}} at the command prompt
and then hit the tab key, the program will guess that you wanted to type
the command name {\tt{read\_data}}, and complete this word for you.  If you
type {\tt{re}} and hit tab, you should hear a beep because there are more
than one possible completions -- hitting tab again will show the three
possibilities {\tt{read\_data}}, {\tt{rename}}, and {\tt{restore}}.  For
the first word on a command-line, {\tt{ifeffit}} will use a restricted set
of commands for tab completion (type 'help' or hit the tab key twice at the
command prompt).  For later words on the command-line, {\tt{ifeffit}} will
use the set of files in the current working directory for tab completion.

In addition to keeping a 'command history' for the current session,
{\tt{ifeffit}} will maintain a recent history of commands between sessions,
and re-load the command history on start-up.  This allows you to scroll
through the commands of earlier sessions.  The commands are kept in memory,
and re-saved into a{\emph{history file}} when {\tt{ifeffit}} exits.  This
file can be used as a record of your session, and as a starting place for
batch command files, scripts, and macros.

The name of the history file and the number of commands saved can be set
with the environmental variables {\tt{IFF\_HISTORY\_FILE}} and
{\tt{IFF\_HISTORY\_LINES}}.  If these variables aren't set, the history
file used will be named {\file{.ifeffit\_hist}} in your home directory and
will contain up to 500 command lines.

When {\tt{ifeffit}} is started, two 'start-up files' will be loaded, if
found.  These are loaded simply as command files, as if you typed their
contents to the command-line.  First, {\file{startup.iff}} in the
installation directory (typically {\file{/usr/local/share/ifeffit}}) is
loaded, and then the file {\file{.ifeffit}} in your home directory is
loaded.  These start-up files are intended for site-wide or personal
definitions for common macros and color-table preferences.  In fact, these
start-up files are not only loaded by the command-line program, but are
loaded by the underlying library, and so the definitions given in these
start-up files can be used in scripts written using the {\ifeffit} library
in perl or python.  This can lead to a portability problem for scripts,
however, and it is not necessarily recommended to rely on definitions in
start-up files in lengthy scripts.

When starting the{\tt{ifeffit}} program, you can give a list of {\ifeffit}
command files and save files that will effectively be {\tt{load}}ed or
{\tt{restore}ed} as necessary before the command-prompt is given to you.
Just to be clear, these are loaded after any start-up files, so definitions
in start-up files can be used in ``{\tt{load}}ed'' files.  If multiple
files are listed they will be {\tt{load}}ed / {\tt{restore}}ed in the order
specified.  To run a command file in 'batch mode', or non-interactively,
you can use the {\tt{-x}} switch, which will load all the files on the
command line and then exit, without ever showing the command prompt.  Note
that {\tt{pause}} statements, which would normally wait for your input are
skipped in this 'batch mode'.  For example,
{\small\begin{verbatim}
  ~> ifeffit  my_file.iff
\end{verbatim}}\noindent
will load {\file{my\_file.iff}} and then give you the command prompt to
continue, while
{\small\begin{verbatim}
  ~> ifeffit -x my_file.iff
\end{verbatim}
}\noindent
will run {\file{my\_file.iff}} and then exit.  If running in batch mode,
you may want to redirect the screen output to a file ({\tt{ifeffit -x
    my\_file.iff > my\_file.out }}), or use the {\tt{-q}} switch
({\tt{ifeffit -q -x my\_file.iff}}) to ``run quietly'', suppressing screen
output altogether.

While in an {\tt{ifeffit}} session, there are a few additional commands
available besides the standard commands of the {\ifeffit} library.  The
standard Unix ``shell commands'' {\tt{ls}}, {\tt{cd}}, {\tt{pwd}}, and
{\tt{more}} work to give information about the current directory and files.
Additional shell commands can be executed by preceding it with a {\tt{!}}
sign, as in
{\small\begin{verbatim}
  Ifeffit> ! gv  ifeffit.ps
\end{verbatim}
}\noindent
which will run {\tt{gv}} (a popular postscript viewer) for you.  Putting an
ampersand {\tt{\&}} at the end of this line will run the job in the
background so that you can continue typing in the {\tt{ifeffit}} shell as
well.  That makes an effective way to edit command files:
{\small\begin{verbatim}
  Ifeffit> ! emacs test_cmnd.iff  &
  Ifeffit> load test_cmnd.iff
\end{verbatim}
}\noindent
Finally, typing 'help' at the {\tt{ifeffit}} prompt will give a
one-line description of the more common {\ifeffit} commands.

\end{document}
%%

